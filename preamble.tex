\documentclass[11pt]{book}
 \usepackage[a4paper, width=160mm, top=25mm, bottom=25mm]{geometry}
 %
 %Fonts
 \usepackage{fontspec}
 \usepackage{xltxtra,xunicode}
 \setmainfont[Mapping=tex-text]{Times New Roman}
 \newfontfamily\greekfont[Script=Greek]{Times New Roman}
 \setsansfont[Scale=MatchLowercase,Mapping=tex-text]{Futura}
 %
 %English - Greek
 \usepackage{polyglossia}
 \setdefaultlanguage[variant=mono]{greek}
 \setotherlanguage[variant=uk]{english}
 \PolyglossiaSetup{greek}{indentfirst=true}
 %
\usepackage{hyperref}
%\hypersetup{
%    colorlinks=true,
%    linkcolor=blue,
%    filecolor=magenta,      
%    urlcolor=cyan,
%    pdfpagemode=FullScreen,
%    }
 \usepackage{caption}
\usepackage{graphicx}

% package to handle line spacing
%\usepackage{setspace}

% HEADINGS
%\usepackage{sectsty} 
%\usepackage[normalem]{ulem} 
%\sectionfont{\rmfamily\mdseries\upshape\Large}
%\subsectionfont{\rmfamily\bfseries\upshape\normalsize} 
%\subsubsectionfont{\rmfamily\mdseries\upshape\normalsize} 

\usepackage{titlesec}

\setcounter{secnumdepth}{4}

\titleformat{\paragraph}
{\normalfont\normalsize\bfseries}{\theparagraph}{1em}{}
\titlespacing*{\paragraph}
{0pt}{3.25ex plus 1ex minus .2ex}{1.5ex plus .2ex}


 %Maths
 \usepackage{amsmath}
 \usepackage{amsfonts}
 \usepackage{amssymb}
 \usepackage{mathrsfs}
 \usepackage{amsthm}
 \usepackage{relsize}
 \usepackage{mathdots}
 \usepackage{mathtools}
 \usepackage{empheq}
 \usepackage{siunitx}
 \usepackage{calc}
 \usepackage{esint}
 \usepackage{esvect}
 \usepackage{tensor}
 \usepackage{physics}
 \usepackage{bm}
 \usepackage{systeme, mathtools}
 \usepackage{stackrel}
 \usepackage{booktabs}
 \usepackage{tabularx}
  \usepackage{polynom}

% package for fancy style headers and footers
\usepackage{fancybox}

% \renewcommand{\baselinestretch}{1.5}
\usepackage{enumitem} %to change enumerate style eg use roman enumeration
\theoremstyle{definition}
\newtheorem*{definition}{Ορισμός}
\newtheorem*{step}{Βήμα}
\newtheorem{exercise}{Άσκηση}
\newtheorem{example}{Παράδειγμα}
\usepackage{colonequals}
\newcommand*{\logeq}{\Leftrightarrow}
\theoremstyle{remark}
\newtheorem*{remark}{Παρατήρηση}
\newtheorem*{application}{Εφαρμογή}
\newtheorem*{solution}{Λύση}
\newtheorem*{claim}{Ισχυρισμός}

\newtheoremstyle{mynote}% name of the style to be used
  {3pt}% measure of space to leave above the theorem. E.g.: 3pt
  {}% measure of space to leave below the theorem. E.g.: 3pt
  {}% name of font to use in the body of the theorem
  {}% measure of space to indent
  {\itshape\bfseries}% name of head font
  {.}% punctuation between head and body
  { }% space after theorem head; " " = normal interword space
  {}% Manually specify head
\theoremstyle{mynote}
\newtheorem*{note}{Σημείωση}

\DeclareMathOperator{\degree}{deg}
\DeclareMathOperator*{\myarrow}{\Rightarrow}

 %%%
\usepackage{makecell} 
 %inserting code
\usepackage{xcolor}
\usepackage{multirow}
\usepackage{multicol} % For multi-column layout
\usepackage{enumitem} % For adjusting item spacing

\usepackage{listings}
% Define custom colors for a light theme
\definecolor{background}{HTML}{fafafa} % Light gray background
\definecolor{keyword}{HTML}{0077aa}    % Keywords: Blue
\definecolor{comment}{HTML}{008800}    % Comments: Green
\definecolor{string}{HTML}{aa5500}     % Strings: Orange/Brown
\definecolor{identifier}{HTML}{333333} % Identifiers: Dark gray
\definecolor{number}{HTML}{990000}     % Numbers: Red

% Define Julia style for listings
\lstdefinelanguage{Julia}{
    keywords={function, end, return, if, else, while, for, in, true, false},
    keywordstyle=\color{keyword}\bfseries,
    identifierstyle=\color{identifier},
    comment=[l]{\#},
    commentstyle=\color{comment}\itshape,
    stringstyle=\color{string},
    morestring=[b]",        % Strings in double quotes
    sensitive=true
}

% Code block settings
\lstset{
    language=Julia,
    basicstyle=\ttfamily\small\color{identifier}, % Default font and text color
    backgroundcolor=\color{background},          % Light background color
    commentstyle=\color{comment}\itshape,        % Comment styling
    keywordstyle=\color{keyword}\bfseries,       % Keywords styling
    stringstyle=\color{string},                  % Strings styling
    numberstyle=\tiny\color{number},             % Line number styling
    stepnumber=1,                                % Line numbers
    numbersep=8pt,                               % Number column spacing
    showstringspaces=false,                      % Hide spaces in strings
    breaklines=true,                             % Line breaking
    frame=single,                                % Frame around code
    rulecolor=\color{background},                % Frame color
    tabsize=4,                                   % Tab spacing
    keepspaces=true
}


\lstdefinestyle{tt}{basicstyle=\small\ttfamily,keywordstyle=\bfseries, rulecolor=\color{background},   
	commentstyle=\color{comment}\itshape, 
	backgroundcolor=\color{background},    
	mathescape=true,
        literate=
               {=}{$\leftarrow{}$}{1}
               {==}{$={}$}{1},
	language=[LaTeX]{TeX}}

\graphicspath{{./Figures/}}

\setlength\parindent{8mm}
%\setlength\parskip{5mm}

% redefine bullet symbols and section style
%\renewcommand\thesection{\arabic{section}.}
%\renewcommand{\labelitemi}{$\blacktriangleright$}
%\renewcommand{\labelitemii}{$\bullet$}

% ---- CUSTOM AMPERSAND
\newcommand{\amper}{{\fontspec[Scale=.95]{Gentium Italic}\selectfont\itshape\&}}


\newcommand{\hRule}{\rule{\linewidth}{0.5pt}}
\newcommand{\HRule}{\rule{\linewidth}{1pt}}
\renewcommand\theadfont{\normalsize}
\renewcommand\lstlistingname{Πρόγραμμα}
\usepackage{import}
\usepackage{float} %lets you prevent LaTeX from repositioning the tables.PUT H e.g.\begin{table}[H]) to make sure it doesn't get repositioned.
\setlength{\parindent}{1em}
%\setlength{\parskip}{1em}
	\makeatletter
\setlength{\@fptop}{0pt}
\makeatother

\newcommand{\olsi}[1]{\,\overline{\!{#1}}}

\makeatletter
\renewcommand{\xRightarrow}[2][]{\ext@arrow 0359\Rightarrowfill@{#1}{#2}}
\makeatother

\makeatletter
\renewcommand{\xLeftrightarrow}[2][]{\ext@arrow 0359\Leftrightarrowfill@{#1}{#2}}
\makeatother