\usepackage{siunitx}
\usepackage{ams math}

% TikZ Packages
\usepackage{tikz}
\usepackage{tkz-euclide}

\usetikzlibrary{angles, quotes, intersections, arrows.meta, babel, backgrounds, decorations, calc, intersections, patterns, positioning}


\newcommand{\drawPoint}[3]{
	\draw[ultra thin, dashed] (0,#2) -- (#1,#2);
	\draw[ultra thin, dashed] (#1,0) -- (#1,#2);
	\filldraw (#1,#2) circle (1pt) coordinate (P#3); 
}

\newcommand{\drawGrid}[4]{
	\draw[very thin, dotted] ($(#3, #4) - (0.5,0.5)$) grid ($(#1,#2) + (0.5,0.5)$);
%Draw axes
    \draw[-latex] ($(#3,0) + (-1,0)$) -- ($(#1,0) + (1,0)$) node[right] {$x$};
    
    \draw[-latex] ($(0,#4)+ (0,-1)$) -- ($(0,#2) + (0,1)$) node[above] {$y$};    
%Draw points in axes    
    \node at (-1.9ex,-1.9ex) {0};
    \foreach \x in {1,...,#1}
	{
		\node at (\x,-2ex) {\x};	
		\draw[thin] (\x, 0) -- (\x, -0.1); % Add a small dash	
	}
		\foreach \y in {1,...,#2}
	{
		\node at (-2ex,\y) {\y};
		\draw[thin] (0, \y) -- (-0.1, \y);			
	}
	
	
	\foreach \xprime in {#3, ..., -1}
	{
		\node at (\xprime,-2ex) {\xprime};	
		\draw[thin] (\xprime, 0) -- (\xprime, -0.1);			
	}
	
	\foreach \yprime in {#4, ..., -1}
	{
		\node at (-2ex, \yprime) {\yprime};		
		\draw[thin] (0, \yprime) -- (-0.1, \yprime);		
	}
}

\newcommand{\drawPosGrid}[2]{
	\draw[very thin, dotted] (-0.5, -0.5) grid ($(#1,#2) + (0.5,0.5)$);
%Draw axes
    \draw[-latex] (0,0) -- ($(#1,0) + (1,0)$) node[right] {$x$};
    
    \draw[-latex] (0,0) -- ($(0,#2) + (0,1)$) node[above] {$y$};    
%Draw points in axes    
    \node at (-1.9ex,-1.9ex) {0};
    \foreach \x in {1,...,#1}
	{
		\node at (\x,-2ex) {\x};	
		\draw[thin] (\x, 0) -- (\x, -0.1); % Add a small dash	
	}
		\foreach \y in {1,...,#2}
	{
		\node at (-2ex,\y) {\y};
		\draw[thin] (0, \y) -- (-0.1, \y);			
	}
	
}


\newcommand{\plainAxis}[2]{
  	\draw[-latex] (0,0) -- (#1,0) coordinate[label=right: $x$] (x);
  	\draw[-latex]  (0,0) -- (0,#2) coordinate[label=above: $y$] (y);
	\filldraw (0,0) circle (1pt) coordinate[label=below left: $O$] (o); 
}

\newcommand{\plainAxisFull}[4]{
  	\draw[-latex] (#3,0) -- (#1,0) coordinate[label=right: $x$] (x);
  	\draw[-latex]  (0,#4) -- (0,#2) coordinate[label=above: $y$] (y);
	\filldraw (0,0) circle (1pt) coordinate[label=below left: $O$] (o); 
}

\newcommand{\drawAxis}[4]{
%    \drawAxis{x}{y}{x'}{y'}
 \node at (-1.9ex,-1.9ex) {0};
\foreach \x in {1,...,#1}
	{
		\node at (\x,-2ex) {\x};	
		\draw[thin] (\x, 0) -- (\x, -0.1); % Add a small dash	
	}
		\foreach \y in {1,...,#2}
	{
		\node at (-2ex,\y) {\y};
		\draw[thin] (0, \y) -- (-0.1, \y);			
	}
	
	
	\foreach \xprime in {#3, ..., -1}
	{
		\node at (\xprime,-2ex) {-\xprime};	
		\draw[thin] (\xprime, 0) -- (\xprime, -0.1);			
	}
	
	\foreach \yprime in {#4, ..., -1}
	{
		\node at (-2ex, \yprime) {-\yprime};		
		\draw[thin] (0, \yprime) -- (-0.1, \yprime);		
	}

    \draw[-latex] (#3,0) -- (#1, 0) coordinate(x) node[right] {$x$}; % Draw x-axis
    \draw[-latex] (0, #4) -- (0, #2) coordinate(y) node[above] {$y$}; % Draw y-axis
}

\newcommand{\drawPosAxis}[2]{
%    \drawAxis{x}{y}{x'}{y'}
 \node at (-1.9ex,-1.9ex) {0};
\foreach \x in {1,...,#1}
	{
		\node at (\x,-2ex) {\x};	
		\draw[thin] (\x, 0) -- (\x, -0.1); % Add a small dash	
	}
		\foreach \y in {1,...,#2}
	{
		\node at (-2ex,\y) {\y};
		\draw[thin] (0, \y) -- (-0.1, \y);			
	}

    \draw[-latex] (0,0) -- (#1, 0) coordinate(x) node[right] {$x$}; % Draw x-axis
    \draw[-latex] (0,0) -- (0, #2) coordinate(y) node[above] {$y$}; % Draw y-axis
}

\newcommand{\drawAxisNoNums}[4]{
%    \drawAxis{x}{y}{x'}{y'}
    \draw[-latex] (#3,0) -- (#1, 0) coordinate(x) node[right] {$x$}; % Draw x-axis
    \draw[-latex] (0, #4) -- (0, #2) coordinate(y) node[above] {$y$}; % Draw y-axis
}

\newcommand{\drawPosAxisNoNums}[2]{
%    \drawAxis{x}{y}{x'}{y'}
    \draw[-latex] (0,0) -- (#1, 0) coordinate(x) node[right] {$x$}; % Draw x-axis
    \draw[-latex] (0,0) -- (0, #2) coordinate(y) node[above] {$y$}; % Draw y-axis
}

\def\centerarc[#1](#2)(#3:#4:#5){ \draw[#1] ($(#2)+({#5*cos(#3)},{#5*sin(#3)})$) arc (#3:#4:#5); }

\newcommand{\drawRectangle}[2]{
	\filldraw[black] (#1, #2) circle (1pt) ;
   	\draw[fill=blue, opacity=0.2] ($(#1,#2) - (0.5,0.5)$) rectangle ($(#1,#2) + (0.5,0.5)$) ;   
}
                
\newcommand{\drawRectangleCol}[4]{
	\filldraw[black] (#1, #2) circle (1pt) ;
   	\draw[fill=#3, opacity=#4] ($(#1,#2) - (0.5,0.5)$) rectangle ($(#1,#2) + (0.5,0.5)$) ;   
}
%
%\tikzset{
%	transition/.style = {
%		semithick,
%		->
%	},
%	every path/.style = {
%		transition, 
%		->, 
%		every node/.style={
%			font=\scriptsize, 
%			align=center
%		}
%	}
%}

