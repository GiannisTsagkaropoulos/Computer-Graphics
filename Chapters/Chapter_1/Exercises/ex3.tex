\begin{exercise}

Να κατασκευαστεί ο αλγόριθμος του Bresenham για ευθεία, για το δεύτερο οκταμόριο.
\end{exercise}

\begin{solution}
	Στο δεύτερο οκταμόριο ισχύει $s >1$, οπότε είναι προτιμότερο η εξίσωση της ευθείας να επιλυθεί ως προς $x$, δηλαδή να εξαχθεί μία σχέση της μορφής $x= f (y)$. Τότε, ο αλγόριθμος του Bresenham λαμβάνει την ακόλουθη μορφή:
	
\begin{lstlisting}[caption={Bresenham Algorithm for 2nd Octant Line Algorithm}]
	x1, y1 = get_coordinates("Give the coordinates of P1 e.g. (1,2):")
	x2, y2 = get_coordinates("Give the coordinates of P2:")
	Dx = x2 - x1
	Dy = y2-y1
	x = x1
	y = y1
	c1 = 2Dx
	error = c1 - Dy
	c2 = error - Dy
	while y <= y2
		if error <0
			error = error + c1
		else
			x = x+1
			error = error + c2
		plot(x,y)
		y = y + 1
		end
	end	
\end{lstlisting}


\end{solution}