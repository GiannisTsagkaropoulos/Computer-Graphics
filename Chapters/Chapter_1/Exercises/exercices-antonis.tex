\documentclass{article}
\usepackage{geometry}
\usepackage{xcolor}
\usepackage{tikz}
\usepackage{amssymb}
\usepackage{mathtools}
\usetikzlibrary{patterns,hobby}
\usetikzlibrary{patterns.meta}
\usepackage{pgfplots}
\usetikzlibrary {shapes.geometric, arrows, arrows.meta}
\usepackage[greek]{babel}
\title{Ασκήσεις Γραφικά με Η/Υ}
\author{Αντώνιος Κομίνατος Γεννατάς \\ ΑΜ : 1112202100077}
\date{\today}
\def\width{6}
\def\hauteur{6}
\def\N{100}
\begin{document}
\maketitle
\section*{Άσκηση 1}
Έστω $\Delta_y = 2$, $\Delta_x = 4$.
Έχουμε ότι $\epsilon_1 = 2\Delta_y- \Delta_x = 0 \geq 0$. Άρα έχουμε ότι φωτίζεται το $pixel$ στην θέση $(4,6)$.  Έχουμε δέιξει ότι ο αναδρομικός τύπος υπολογισμού των $\epsilon_i$ είναι:
\begin{center}
    \begin{tabular}{|c|c|c|}
     \hline
        $\epsilon_i < 0$ & $\epsilon_{i+1} = \epsilon_i + 2\Delta_y$ & $y_{i+1} = y_i$\\
     \hline
        $\epsilon_i \geq	 0$ & $ \epsilon_{i+1} = \epsilon_i + 2\Delta_y- 2\Delta_x $& $y_{i+1} = y_i + 1$ \\
     \hline
\end{tabular}
\end{center}
Άρα έχουμε.
\[
    \epsilon_{2} = \epsilon_1 + 2\Delta_y- 2\Delta_x = -4 < 0.
\]
Άρα το επόμενο $pixel$ το οποίο φωτίζεται είναι το $(5,6)$.
\[
    \epsilon_{3} = \epsilon_2 + 2\Delta_y = 0 \geq 0.
\]
Άρα το επόμενο $pixel$ το οποίο φωτίζεται είναι το $(6,7)$.
\[
    \epsilon_{4} = \epsilon_3 + 2\Delta_y- 2\Delta_x = -4 < 0.
\]
Άρα το επόμενο $pixel$ το οποίο φωτίζεται είναι το $(7,7)$.
\begin{align*}
    \begin{tikzpicture}
        \draw[step=1cm, line width=0.5mm, gray!10!gray] (0,0) grid (6,3);
        \foreach \x in {0.5,1.5,2.5,3.5,4.5,5.5} {
                \node [fill= gray, fill opacity=0.1, draw=none, thick, minimum size=1cm] at (\x,0.5);
                \node [fill= gray, fill opacity=0.1, draw=none, thick, minimum size=1cm] at (\x,1.5);
                \node [fill= gray, fill opacity=0.1, draw=none, thick, minimum size=1cm] at (\x,2.5);
                }
        \draw (0.5,0.5) -- (4.5,2.5)
        \draw [pattern={Lines[angle=45,distance=4pt,]},pattern color=orange]  (0.5,0.5) circle [radius=0.5];
        \draw [pattern={Lines[angle=45,distance=4pt,]},pattern color=orange]  (1.5,1.5) circle [radius=0.5];
        \draw [pattern={Lines[angle=45,distance=4pt,]},pattern color=orange]  (2.5,1.5) circle [radius=0.5];
        \draw [pattern={Lines[angle=45,distance=4pt,]},pattern color=orange]  (3.5,2.5) circle [radius=0.5];
        \draw [pattern={Lines[angle=45,distance=4pt,]},pattern color=orange]  (4.5,2.5) circle [radius=0.5];
        \node[anchor= north] at (0.5,0) {\large $3$};
        \node[anchor= north] at (1.5,0) {\large $4$};
        \node[anchor= north] at (2.5,0) {\large $5$};
        \node[anchor= north] at (3.5,0) {\large $6$};
        \node[anchor= north] at (4.5,0) {\large $7$};
        \node[anchor= north] at (5.5,0) {\large $8$};
        \node[anchor= east] at (0,0.5) {\large $5$};
        \node[anchor= east] at (0,1.5) {\large $6$};
        \node[anchor= east] at (0,2.5) {\large $7$};
    \end{tikzpicture}
\end{align*}
Έστω $\Delta_y = 1$, $\Delta_x = 5$.
Έχουμε ότι $\epsilon_1 = 2\Delta_y- \Delta_x = -3 < 0$. Άρα έχουμε ότι φωτίζεται το $pixel$ στην θέση $(4,5)$.
\[
    \epsilon_{2} = \epsilon_1 + 2\Delta_y = -1 < 0.
\]
Άρα το επόμενο $pixel$ το οποίο φωτίζεται είναι το $(5,5)$.
\[
    \epsilon_{3} = \epsilon_2 + 2\Delta_y = 1 > 0.
\]
Άρα το επόμενο $pixel$ το οποίο φωτίζεται είναι το $(6,6)$.
\[
    \epsilon_{4} = \epsilon_3 + 2\Delta_y- 2\Delta_x = -7 < 0.
\]
Άρα το επόμενο $pixel$ το οποίο φωτίζεται είναι το $(7,6)$.
\[
    \epsilon_{5} = \epsilon_4 + 2\Delta_y- 2\Delta_x = -15 < 0.
\]
Άρα το επόμενο $pixel$ το οποίο φωτίζεται είναι το $(8,6)$.
\begin{align*}    
    \begin{tikzpicture}
        \draw[step=1cm, line width=0.5mm, gray!10!gray] (0,0) grid (6,3);
        \foreach \x in {0.5,1.5,2.5,3.5,4.5,5.5} {
            \node [fill= gray, fill opacity=0.1, draw=none, thick, minimum size=1cm] at (\x,0.5);
            \node [fill= gray, fill opacity=0.1, draw=none, thick, minimum size=1cm] at (\x,1.5);
            \node [fill= gray, fill opacity=0.1, draw=none, thick, minimum size=1cm] at (\x,2.5);
                }
        \draw (0.5,0.5) -- (5.5,1.5)
        \draw [pattern={Lines[angle=45,distance=4pt,]},pattern color=orange]  (0.5,0.5) circle [radius=0.5];
        \draw [pattern={Lines[angle=45,distance=4pt,]},pattern color=orange]  (1.5,0.5) circle [radius=0.5];
        \draw [pattern={Lines[angle=45,distance=4pt,]},pattern color=orange]  (2.5,0.5) circle [radius=0.5];
        \draw [pattern={Lines[angle=45,distance=4pt,]},pattern color=orange]  (3.5,1.5) circle [radius=0.5];
        \draw [pattern={Lines[angle=45,distance=4pt,]},pattern color=orange]  (4.5,1.5) circle [radius=0.5];
        \draw [pattern={Lines[angle=45,distance=4pt,]},pattern color=orange]  (5.5,1.5) circle [radius=0.5];
        \node[anchor= north] at (0.5,0) {\large $3$};
        \node[anchor= north] at (1.5,0) {\large $4$};
        \node[anchor= north] at (2.5,0) {\large $5$};
        \node[anchor= north] at (3.5,0) {\large $6$};
        \node[anchor= north] at (4.5,0) {\large $7$};
        \node[anchor= north] at (5.5,0) {\large $8$};
        \node[anchor= east] at (0,0.5) {\large $5$};
        \node[anchor= east] at (0,1.5) {\large $6$};
        \node[anchor= east] at (0,2.5) {\large $7$};
    \end{tikzpicture}
\end{align*}
\section*{Άσκηση 2}
\subsection*{$i)$}
Άρχικα έχουμε ότι φωτίζεται το $pixel$ στην θέση $(0,5)$. Έπειτα υπολογίζουμε την ποσότητα:
\[
\epsilon_1 = 3-2r \overset{r=5}{=} -7 < 0
\]
Άρα το επόμενο $pixel$ το οποίο φωτίζεται είναι το $(1,5)$. Έχουμε δέιξει ότι ο αναδρομικός τύπος υπολογισμού των $\epsilon_i$ είναι: 
\begin{center}
    \begin{tabular}{|c|c|c|}
     \hline
        $\epsilon_i < 0$ & $\epsilon_{i+1} = \epsilon_i + 4(x_i + 1) + 2$ & $y_{i+1} = y_i$ \\
     \hline
        $\epsilon_i \geq	 0$ & $ \epsilon_{i+1} = \epsilon_i + 4(x_i + 1) + 2 - 4(y_i - 1)$& $y_{i+1} = y_i -1$\\ 
     \hline
\end{tabular}
\end{center}
Άρα έχουμε.
\[
    \epsilon_{2} = \epsilon_1 + 4(x_1 + 1) + 2 = -7 + 6 = -1 < 0  
\]   
Άρα το επόμενο $pixel$ το οποίο φωτίζεται είναι το $(2,5)$.
\[
    \epsilon_{3} = \epsilon_2 + 4(x_2 + 1) + 2 = -1 + 14 = 13 > 0
\]
Άρα το επόμενο $pixel$ το οποίο φωτίζεται είναι το $(3,4)$. Το επόμενο $pixel$ που θα φωτιστεί βρίσκεται στο $1^{\underset{=}{o}}$ οκταμόριο άρα αρκει να πούμε ότι $(x_5 ,y_5 ) = (y_4 , x_4)$
Άρα το επόμενο $pixel$ το οποίο φωτίζεται είναι το $(4,3)$.
\begin{align*}
    \begin{tikzpicture}
        \draw[step=1cm, line width=0.5mm, gray!10!gray] (0,0) grid (5,3);
        \foreach \x in {0.5,1.5,2.5,3.5,4.5} {
                \node [fill= gray, fill opacity=0.1, draw=none, thick, minimum size=1cm] at (\x,0.5);
                \node [fill= gray, fill opacity=0.1, draw=none, thick, minimum size=1cm] at (\x,1.5);
                \node [fill= gray, fill opacity=0.1, draw=none, thick, minimum size=1cm] at (\x,2.5);
                }
        \draw [pattern={Lines[angle=45,distance=4pt,]},pattern color=orange]  (0.5,2.5) circle [radius=0.5];
        \draw [pattern={Lines[angle=45,distance=4pt,]},pattern color=orange]  (1.5,2.5) circle [radius=0.5];
        \draw [pattern={Lines[angle=45,distance=4pt,]},pattern color=orange]  (2.5,2.5) circle [radius=0.5];
        \draw [pattern={Lines[angle=45,distance=4pt,]},pattern color=orange]  (3.5,1.5) circle [radius=0.5];
        \draw [pattern={Lines[angle=45,distance=4pt,]},pattern color=orange]  (4.5,0.5) circle [radius=0.5];
        \node[anchor= north] at (0.5,0) {\large $0$};
        \node[anchor= north] at (1.5,0) {\large $1$};
        \node[anchor= north] at (2.5,0) {\large $2$};
        \node[anchor= north] at (3.5,0) {\large $3$};
        \node[anchor= north] at (4.5,0) {\large $4$};
        \node[anchor= east] at (0,0.5) {\large $3$};
        \node[anchor= east] at (0,1.5) {\large $4$};
        \node[anchor= east] at (0,2.5) {\large $5$};
    \end{tikzpicture}
\end{align*}
\subsection*{$ii)$}
Τα $pixel$ που θα φωτιστούν στο $3^{\underset{=}{o}}$ οκταμόριο για τον κύκλο με κέντρο $(7,3)$ και ακτίνα $r=5$ είναι τα ίδια που βρήκαμε προηγουμένος με τους εξής μετασχηματισμούς:
\[
(x^{'}_i , y^{'}_i) = (-(x_i + 7), y_i + 3)
\]
Άρα τα 5 πρώτα $pixel$ που θα χρωματιστούν είναι τα: 
$\begin{cases}
    (-7,8)\\
    (-8,8)\\
    (-9,8)\\
    (-10,7)\\
    (-11,6)\\
\end{cases}$
\begin{align*}
    \begin{tikzpicture}
        \draw[step=1cm, line width=0.5mm, gray!10!gray] (0,0) grid (5,3);
        \foreach \x in {0.5,1.5,2.5,3.5,4.5} {
                \node [fill= gray, fill opacity=0.1, draw=none, thick, minimum size=1cm] at (\x,0.5);
                \node [fill= gray, fill opacity=0.1, draw=none, thick, minimum size=1cm] at (\x,1.5);
                \node [fill= gray, fill opacity=0.1, draw=none, thick, minimum size=1cm] at (\x,2.5);
                }
        \draw [pattern={Lines[angle=45,distance=4pt,]},pattern color=orange]  (4.5,2.5) circle [radius=0.5];
        \draw [pattern={Lines[angle=45,distance=4pt,]},pattern color=orange]  (3.5,2.5) circle [radius=0.5];
        \draw [pattern={Lines[angle=45,distance=4pt,]},pattern color=orange]  (2.5,2.5) circle [radius=0.5];
        \draw [pattern={Lines[angle=45,distance=4pt,]},pattern color=orange]  (1.5,1.5) circle [radius=0.5];
        \draw [pattern={Lines[angle=45,distance=4pt,]},pattern color=orange]  (0.5,0.5) circle [radius=0.5];
        \node[anchor= north] at (0.5,0) {\large $-11$};
        \node[anchor= north] at (1.5,0) {\large $-10$};
        \node[anchor= north] at (2.5,0) {\large $-9$};
        \node[anchor= north] at (3.5,0) {\large $-8$};
        \node[anchor= north] at (4.5,0) {\large $-7$};
        \node[anchor= east] at (0,0.5) {\large $6$};
        \node[anchor= east] at (0,1.5) {\large $7$};
        \node[anchor= east] at (0,2.5) {\large $8$};
    \end{tikzpicture}
\end{align*}
\section*{Άσκηση 3}
\begin{align*}
    \begin{minipage}{0.4\textwidth}
        \begin{tikzpicture}
            \draw[step=1.5cm, line width=0.5mm, gray!10!gray] (0,0) grid (4.5,4.5);
            \draw[step=1.5cm, line width=0.6mm, black!10!black] (1.5,0) grid (4.5,3);
            \draw [pattern={Lines[angle=45,distance=4pt,]},pattern color=orange]  (3.75,0.75) circle [radius=0.75];
            \draw (3.7,0) arc [x radius= 8, y radius= 8.5, start angle=0, end angle=32];
            \foreach \x in {0.75,2.25,3.75} {
                \node [fill= gray, fill opacity=0.1, draw=none, thick, minimum size=1.5cm] at (\x,0.75);
                \node [fill= gray, fill opacity=0.1, draw=none, thick, minimum size=1.5cm] at (\x,2.25);
                \node [fill= gray, fill opacity=0.1, draw=none, thick, minimum size=1.5cm] at (\x,3.75);
                }
            \node[anchor=east] at (0,0.75) {\Large $0$};
            \node[anchor=east] at (0,2.25) {\Large $1$};
            \node[anchor=east] at (0,3.75) {\Large $2$};
            \node[anchor=north] at (0.75,0) {\Large $3$};
            \node[anchor=north] at (2.25,0) {\Large $4$};
            \node[anchor=north] at (3.75,0) {\Large $5$};
            \node at (2.25,2.25) {\large $C$};
            \node at (3.75,2.25) {\large $D$};
            \node[anchor=south] at (2.7,3) {\large $M$};
            \node[anchor=south] at (3.4,3) {\large $T$};
            \node[anchor=south] at (3.3,4.5) {\large \color{white} $T$};
            \node[anchor=south] at (2.25,4.5) {\large \color{white} $ t$};
        \end{tikzpicture}
    \end{minipage}
    \begin{minipage}{0.4\textwidth}
        \begin{tikzpicture}
            \draw[step=1.5cm, line width=0.5mm, gray!10!gray] (0,0) grid (4.5,4.5);
            \draw[step=1.5cm, line width=0.6mm, black!10!black] (1.5,1.5) grid (4.5,4.5);
            \draw [pattern={Lines[angle=45,distance=4pt,]},pattern color=orange]  (3.75,0.75) circle [radius=0.75];
            \draw [pattern={Lines[angle=45,distance=4pt,]},pattern color=orange]  (3.75,2.25) circle [radius=0.75];
            \draw (3.7,0) arc [x radius= 8, y radius= 8.5, start angle=0, end angle=32];
            \foreach \x in {0.75,2.25,3.75} {
                \node [fill= gray, fill opacity=0.1, draw=none, thick, minimum size=1.5cm] at (\x,0.75);
                \node [fill= gray, fill opacity=0.1, draw=none, thick, minimum size=1.5cm] at (\x,2.25);
                \node [fill= gray, fill opacity=0.1, draw=none, thick, minimum size=1.5cm] at (\x,3.75);
                }
            \node[anchor=east] at (0,0.75) {\Large $0$};
            \node[anchor=east] at (0,2.25) {\Large $1$};
            \node[anchor=east] at (0,3.75) {\Large $2$};
            \node[anchor=north] at (0.75,0) {\Large $3$};
            \node[anchor=north] at (2.25,0) {\Large $4$};
            \node[anchor=north] at (3.75,0) {\Large $5$};
            \node at (2.25,3.75) {\large $C$};
            \node at (3.75,3.75) {\large $D$};
            \node[anchor=south] at (3.3,4.5) {\large $M$};
            \node[anchor=south] at (2.25,4.5) {\large $T$};
        \end{tikzpicture}
    \end{minipage}
    \begin{minipage}{0.4\textwidth}
        \begin{tikzpicture}
            \draw[step=1.5cm, line width=0.5mm, gray!10!gray] (0,0) grid (4.5,4.5);
            \draw [pattern={Lines[angle=45,distance=4pt,]},pattern color=orange]  (3.75,0.75) circle [radius=0.75];
            \draw [pattern={Lines[angle=45,distance=4pt,]},pattern color=orange]  (3.75,2.25) circle [radius=0.75];
            \draw [pattern={Lines[angle=45,distance=4pt,]},pattern color=orange]  (2.25,3.75) circle [radius=0.75];
            \draw (3.7,0) arc [x radius= 8, y radius= 8.5, start angle=0, end angle=32];
            \foreach \x in {0.75,2.25,3.75} {
                \node [fill= gray, fill opacity=0.1, draw=none, thick, minimum size=1.5cm] at (\x,0.75);
                \node [fill= gray, fill opacity=0.1, draw=none, thick, minimum size=1.5cm] at (\x,2.25);
                \node [fill= gray, fill opacity=0.1, draw=none, thick, minimum size=1.5cm] at (\x,3.75);
                }
            \node[anchor=east] at (0,0.75) {\Large $0$};
            \node[anchor=east] at (0,2.25) {\Large $1$};
            \node[anchor=east] at (0,3.75) {\Large $2$};
            \node[anchor=north] at (0.75,0) {\Large $3$};
            \node[anchor=north] at (2.25,0) {\Large $4$};
            \node[anchor=north] at (3.75,0) {\Large $5$};
            \node[anchor=south] at (3.3,4.5) {\large \color{white} $T$};
            \node[anchor=south] at (2.25,4.5) {\large \color{white} $ t$};
        \end{tikzpicture}
    \end{minipage}
\end{align*}
\begin{align*}
    f_{mid1,1} = a^2 - b^2 a + \frac{b^2}{4} \overset{a=5,b=3}{=} 25 - 45 + \frac{9}{4}= -17.75 < 0 
\end{align*} 
Άρα φωτίζεται το $pixel$ στην θέση $(5,1)$. Έχουμε δείξει ότι ο αναδρομικός τύπος του $f_{mid1,i}$ είναι:
\begin{center}
    \begin{tabular}{ |c|c|c| } 
        \hline
        $f_{mid1,i}<0$ &$ f_{mid1,i} +2a^2 (y_{i}+1) + a^2$& Επιλογή σημείου $D$\\
        \hline
        $f_{mid1,i} \geq 0$ &$ f_{mid1,i} +2a^2 (y_{i}+1) + a^2 -2b^2(x_{i}-1)$& Επιλογή σημείου $C$\\
        \hline  
    \end{tabular}
\end{center}
Άρα έχουμε ότι:
\begin{align*}
     f_{mid1,2} = f_{mid1,1} + 2a^2 (y_{i}+1) + a^2 \overset{a=5}{=} -17.75 + 75 = 57.25 > 0 
\end{align*}
Άρα φωτίζεται το $pixel$ στην θέση $(4,2)$. Ελέγχουμε αν έχουμε περάσει στην περιοχή 2:
\begin{align*}
     2b^2x > 2a^2y \overset{a=5,b=3}{\Rightarrow} 72\ngtr100 
\end{align*}
άρα περάσαμε στην περιοχή 2. Υπολογίζουμε το $f_{mid2,1}$:
\begin{align*}
    f_{mid2,1} = f_ {mid1,2} - \frac {(2b^ {2}x_ {3}+2a^ {2}y_ {3})}{2}  +0.75(  b^ {2}  -  a^ {2} ) = 57.25 - 86 -12 = -40.75 < 0 
\end{align*}
Άρα φωτίζεται το $pixel$ στην θέση $(3,3)$. Έχουμε δείξει ότι ο αναδρομικός τύπος του $f_{mid2,i}$ είναι:
\begin{center}
    \begin{tabular}{ |c|c| } 
        \hline
        $f_{mid2,i}<0$ &$ f_{mid2,i} -2b^2 (x_{i}-1) + b^2 + 2a^2 (y_{i}+1)$\\
        \hline
        $f_{mid2,i} \geq 0$ &$ f_{mid2,i} -2b^2 (x_{i}-1) + b^2 $\\
        \hline  
    \end{tabular}
\end{center}
Άρα έχουμε ότι:
\begin{align*}
   f_{mid2,2} = f_{mid2,1} -2b^2 (x_{3}-1) + b^2 = -40.75 - 32 +9 + 150 = 86.25 > 0 
\end{align*}
Άρα φωτίζεται το $pixel$ στην θέση $(3,2)$.
\begin{align*}
    \begin{minipage}{0.4\textwidth}
        \begin{tikzpicture}
            \draw[step=1.5cm, line width=0.5mm, gray!10!gray] (0,0) grid (4.5,4.5);
            \draw[step=1.5cm, line width=0.6mm, black!10!black] (1.5,0) grid (4.5,3);
            \draw [pattern={Lines[angle=45,distance=4pt,]},pattern color=orange]  (3.75,0.75) circle [radius=0.75];
            \draw (3.7,0) arc [x radius= 4.2, y radius= 4.5, start angle=28, end angle=90];
            \foreach \x in {0.75,2.25,3.75} {
                \node [fill= gray, fill opacity=0.1, draw=none, thick, minimum size=1.5cm] at (\x,0.75);
                \node [fill= gray, fill opacity=0.1, draw=none, thick, minimum size=1.5cm] at (\x,2.25);
                \node [fill= gray, fill opacity=0.1, draw=none, thick, minimum size=1.5cm] at (\x,3.75);
                }
            \node[anchor=east] at (0,0.75) {\Large $2$};
            \node[anchor=east] at (0,2.25) {\Large $3$};
            \node[anchor=east] at (0,3.75) {\Large $4$};
            \node[anchor=north] at (0.75,0) {\Large $2$};
            \node[anchor=north] at (2.25,0) {\Large $3$};
            \node[anchor=north] at (3.75,0) {\Large $4$};
            \node at (2.25,2.25) {\large $C$};
            \node at (2.25,0.75) {\large $B$};
            \node[anchor=east] at (1.5,1.75) {\large $M$};
            \node[anchor=east] at (1.5,2.5) {\large $T$};
            \node[anchor=south] at (3.3,4.5) {\large \color{white} $T$};
            \node[anchor=south] at (2.25,4.5) {\large \color{white} $ t$};
        \end{tikzpicture}
    \end{minipage}
    \begin{minipage}{0.4\textwidth}
        \begin{tikzpicture}
            \draw[step=1.5cm, line width=0.5mm, gray!10!gray] (0,0) grid (4.5,4.5);
            \draw[step=1.5cm, line width=0.6mm, black!10!black] (0,1.5) grid (3,4.5);
            \draw [pattern={Lines[angle=45,distance=4pt,]},pattern color=orange]  (3.75,0.75) circle [radius=0.75];
            \draw [pattern={Lines[angle=45,distance=4pt,]},pattern color=orange]  (2.25,2.25) circle [radius=0.75];
            \draw (3.7,0) arc [x radius= 4.2, y radius= 4.5, start angle=28, end angle=90];
            \foreach \x in {0.75,2.25,3.75} {
                \node [fill= gray, fill opacity=0.1, draw=none, thick, minimum size=1.5cm] at (\x,0.75);
                \node [fill= gray, fill opacity=0.1, draw=none, thick, minimum size=1.5cm] at (\x,2.25);
                \node [fill= gray, fill opacity=0.1, draw=none, thick, minimum size=1.5cm] at (\x,3.75);
                }
            \node[anchor=east] at (0,0.75) {\Large $2$};
            \node[anchor=east] at (0,2.25) {\Large $3$};
            \node[anchor=east] at (0,3.75) {\Large $4$};
            \node[anchor=north] at (0.75,0) {\Large $2$};
            \node[anchor=north] at (2.25,0) {\Large $3$};
            \node[anchor=north] at (3.75,0) {\Large $4$};
            \node at (0.75,3.75) {\large $C$};
            \node at (0.75,2.25) {\large $B$};
            \node[anchor=east] at (0,3) {\large $M$};
            \node[anchor=west] at (0,2.6) {\large $T$};
            \node[anchor=south] at (3.3,4.5) {\large \color{white} $T$};
            \node[anchor=south] at (2.25,4.5) {\large \color{white} $ t$};
        \end{tikzpicture}
    \end{minipage}
    \begin{minipage}{0.4\textwidth}
        \begin{tikzpicture}
            \draw[step=1.5cm, line width=0.5mm, gray!10!gray] (0,0) grid (4.5,4.5);
            \draw [pattern={Lines[angle=45,distance=4pt,]},pattern color=orange]  (3.75,0.75) circle [radius=0.75];
            \draw [pattern={Lines[angle=45,distance=4pt,]},pattern color=orange]  (2.25,2.25) circle [radius=0.75];
            \draw [pattern={Lines[angle=45,distance=4pt,]},pattern color=orange]  (0.75,2.25) circle [radius=0.75];
            \draw (3.7,0) arc [x radius= 4.2, y radius= 4.5, start angle=28, end angle=90];
            \foreach \x in {0.75,2.25,3.75} {
                \node [fill= gray, fill opacity=0.1, draw=none, thick, minimum size=1.5cm] at (\x,0.75);
                \node [fill= gray, fill opacity=0.1, draw=none, thick, minimum size=1.5cm] at (\x,2.25);
                \node [fill= gray, fill opacity=0.1, draw=none, thick, minimum size=1.5cm] at (\x,3.75);
                }
            \node[anchor=east] at (0,0.75) {\Large $2$};
            \node[anchor=east] at (0,2.25) {\Large $3$};
            \node[anchor=east] at (0,3.75) {\Large $4$};
            \node[anchor=north] at (0.75,0) {\Large $2$};
            \node[anchor=north] at (2.25,0) {\Large $3$};
            \node[anchor=north] at (3.75,0) {\Large $4$};
            \node[anchor=south] at (3.3,4.5) {\large \color{white} $T$};
            \node[anchor=south] at (2.25,4.5) {\large \color{white} $ t$};
        \end{tikzpicture}
    \end{minipage}
\end{align*}
Το συνολικό σχήμα που φωτίστηκε είναι το παρακάτω.
\begin{align*}
    \begin{tikzpicture}[x=1cm, y=1cm]
        \draw[step=1.5cm, line width=0.5mm, gray!10!gray] (0,0) grid (6,6);      
        \draw [pattern={Lines[angle=45,distance=4pt,]},pattern color=orange]  (5.25,0.75) circle [radius=0.75];
        \draw [pattern={Lines[angle=45,distance=4pt,]},pattern color=orange]  (5.25,2.25) circle [radius=0.75];
        \draw [pattern={Lines[angle=45,distance=4pt,]},pattern color=orange]  (3.75,3.75) circle [radius=0.75];
        \draw [pattern={Lines[angle=45,distance=4pt,]},pattern color=orange]  (2.25,5.25) circle [radius=0.75];
        \draw [pattern={Lines[angle=45,distance=4pt,]},pattern color=orange]  (0.75,5.25) circle [radius=0.75];
        \node[anchor=east] at (0,0.75) {\Large $0$};
        \node[anchor=east] at (0,2.25) {\Large $1$};
        \node[anchor=east] at (0,3.75) {\Large $2$};
        \node[anchor=east] at (0,5.25) {\Large $3$};
        \node[anchor=north] at (0.75,0) {\Large $2$};
        \node[anchor=north] at (2.25,0) {\Large $3$};
        \node[anchor=north] at (3.75,0) {\Large $4$};
        \node[anchor=north] at (5.25,0) {\Large $5$};
        \node at (3.75,1.5) {\Huge \color{blue}$1$};
        \node at (1.5,3.75) {\Huge \color{blue}$2$};
        \draw[scale=0.5, domain=0:12, smooth, variable=\x, blue] plot ({\x}, {\x});
        \foreach \x in {0.75,2.25,3.75,5.25} {
                \node [fill= gray, fill opacity=0.1, draw=none, thick, minimum size=1.5cm] at (\x,0.75);
                \node [fill= gray, fill opacity=0.1, draw=none, thick, minimum size=1.5cm] at (\x,2.25);
                \node [fill= gray, fill opacity=0.1, draw=none, thick, minimum size=1.5cm] at (\x,3.75);
                \node [fill= gray, fill opacity=0.1, draw=none, thick, minimum size=1.5cm] at (\x,5.25);
                }
    \end{tikzpicture}
\end{align*}
\section*{Άσκηση 4}
΄Εστω ότι έχει φωτισθεί το $pixel$ $(x_{i}, y_{i})$. Επειδή βρισκόμαστε στην περιοχή 1 το επόμενο προς φωτισμό $pixel$ θα είναι το $(x_{i-1}, y_{i+1})$ ή το $(x_{i-1}, y_{i})$ .
\begin{align*}
    \begin{tikzpicture}[x=1cm, y=1cm]
    \draw[step=3cm, line width=0.5mm, gray!10!gray] (0,0) grid (\width,\hauteur); 
      \node[anchor=north] at (1.5,0) {\Large $x_{i-1}$};
      \node at (1.5,4.5) {\Huge $C$};
      \node at (1.5,1.5) {\Huge $B$};
      \node at (4.5,4.5) {\Huge $D$};
      \node[anchor=east] at (0,3) {\huge $M$};
      \node[anchor=east] at (0,1.5) {\Large $y_{i}$};
      \node[anchor=east] at (0,4.5) {\Large $y_{i+1}$};
      \node[anchor=north] at (4.5,0) {\Large $ x_{i}$};
      \draw [pattern={Lines[angle=45,distance=4pt,]},pattern color=orange]  (4.5,1.5) circle [radius=1.5];
      \draw (4.8,0) arc [x radius=4.8, y radius=4.2, start angle=0, end angle=90];
      \draw (4.8,0) arc [x radius=4.8, y radius=2.7, start angle=0, end angle=90];
      \foreach \x in {1.5,4.5} {
            \node [fill= gray, fill opacity=0.1, draw=none, thick, minimum size=3cm] at (\x,4.5);
            \node [fill= gray, fill opacity=0.1, draw=none, thick, minimum size=3cm] at (\x,1.5);
            }
      \draw [decorate,
            decoration = {brace,mirror}] (0.1,3.02) --  (0.1,4.18)
            node[pos=0.5,right=2pt,black]{$e_i$};      
    \end{tikzpicture}
\end{align*}
\begin{gather*}
    f_{mid2,i+1} = b^2 (x_{i}-1-1)^2 + a^2 (y_{i+1}+\frac{1}{2})^2 -a^2 b^2 = 
    \\ = \underbrace{b^2 (x_{i}-1)^2 + a^2 (y_{i}+\frac{1}{2})^2 -a^2 b^2}_\text{$f_{mid2,i}$} -a^2 (y_{i}+\frac{1}{2})^2 -2b^2 x_{i-1} + b^2 + a^2 y^2_{i+1} + a^2 y_{i+1} + \frac{a^2}{4} = 
    \\ = f_{mid2,i} - a^2 (y_{i}+\frac{1}{2})^2 -2b^2 (x_{i}-1) + b^2 + a^2 y^2_{i+1} + a^2 y_{i+1} + \frac{a^2}{4}= 
    \\ = f_{mid2,i} - a^2 y^2_{i} - a^2 y_{i} - \frac{a^2}{4} -2b^2 (x_{i}-1) + b^2 + a^2 y^2_{i+1} + a^2 y_{i+1} + \frac{a^2}{4} = 
    \\ = f_{mid2,i} + a^2 (y^2_{i+1} - y^2_{i}) + a^2 (y_{i+1} - y_{i}) -2b^2 (x_{i}-1) + b^2.
\end{gather*}
\begin{center}
    \begin{tabular}{ |c|c| } 
        \hline
        $f_{mid2,i}<0$ &$ f_{mid2,i} -2b^2 (x_{i}-1) + b^2 + 2a^2 y_{i+1}$\\
        \hline
        $f_{mid2,i} \geq 0$ &$ f_{mid2,i} -2b^2 (x_{i}-1) + b^2 $\\
        \hline  
    \end{tabular}
\end{center}
Άρα έχουμε ότι ο αναδρομικός τύπος του $f_{mid2,i}$ είναι:
\begin{gather*}
   f_{mid2,i+1} = f_{mid2,i} + a^2 (y^2_{i+1} - y^2_{i}) + a^2 (y_{i+1} - y_{i}) -2b^2 (x_{i}-1) + b^2.
\end{gather*}
\end{document}
