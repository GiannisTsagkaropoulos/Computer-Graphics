\begin{exercise}
Δείξτε ότι κάθε μεγιστικό ως προς το πλήθος ακμών εξωεπίπεδο γράφημα με $n \geq 3$ κορυφές έχει ακριβώς $n - 1$ όψεις.
\end{exercise}

\begin{solution}
Έστω $G$ μεγιστικό ως προς το πλήθος ακμών εξωεπίπεδο γράφημα με $n \geq 3$ κορυφές.

\begin{claim}
	Αν $G$ μεγιστικό ως προς το πλήθος ακμών εξωεπίπεδο γράφημα με $n \geq 3$ κορυφές, τότε αυτό είναι συνεκτικό.
\end{claim}

Πράγματι, έστω προς άτοπο, ότι $G$ όχι συνεκτικό και έστω 2 συνεκτικές συνιστώσες του. Τότε ενώνοντας 2 κορυφές που ανήκουν στις διαφορετικές συνεκτικές συνιστώσες προκύπτει εξωεπίπεδο γράφημα, άτοπο, καθώς το $G$ είναι μεγιστικό ως προς το πλήθος ακμών του.

Εφόσον λοιπόν το $G$ είναι συνεκτικό, ισχύει το θεώρημα του Euler και άρα 
\[
	m+2=r+n \Rightarrow r = m+2-n
\] 

Θέλουμε να δείξουμε ότι για το $G$ ισχύει
\begin{equation}
	r = n-1 \Longleftrightarrow m +2 - n = n-1 \Longleftrightarrow m = 2n-3.
	\label{eq:1}	
\end{equation}



Αρκεί να δείξουμε, λοιπόν, αν $G$ μεγιστικό ως προς το πλήθος ακμών εξωεπίπεδο γράφημα με $n \geq 3$ κορυφές, τότε $m = 2n-3$.
		Έστω $n \geq 3$. Γνωρίζουμε ότι κάθε εξωεπίπεδο γράφημα με $n$ κορυφές έχει το πολύ $2n-3$ ακμές. Αν βρούμε εξωεπίπεδο γράφημα $n$ κορυφές και ακριβώς  $2n-3$ ακμές, τότε αυτό θα είναι μεγιστικό ως προς το πλήθος ακμών εξωεπίπεδο. 
		
		Θα κατασκευάσουμε αυτό το γράφημα και \textit{θα δείξουμε ότι} για αυτό ισχύει $m=2n-3$ με επαγωγή. Θεωρώ $n$ κορυφές. Τότε παίρνω το $C_n$. Επιλέγω μία κορυφή του κύκλου, έστω $u$ και την ενώνω με όλες τις κορυφές του κύκλου. Τέλος, αφαιρώ τις δύο ακμές που την ενώνουν με τις δύο κορυφές που ήταν οι γειτονικές της στον κύκλο. Σχηματικά έχουμε: 
		
%%%%%%%%%%%%%%%%%%%%%		
\begin{figure*}[h!]
	\includegraphics{ex4circle.pdf}
	\includegraphics{ex4maximal.pdf}
	\caption{Παράδειγμα μεγιστικού ως προς πλήθος ακμών εξωεπιπέδου για $n=12$}
\end{figure*}	
%%%%%%%%%%%%%%%%%%%%%	
 \begin{proof}
 	\textit{Βάση επαγωγής:} Έστω $n=3$. Τότε το αντίστοιχο γράφημα είναι το $K_3$ το οποίο έχει $3=2\cdot 3 - 3$ ακμές.
 	
 	
 	\textit{Επαγωγική υπόθεση:} Έστω ότι το γράφημα $G$ που περιγράψαμε παραπάνω  για $3,4, \ldots, n-1$ κορυφές έχει $2n(G)-3$ ακμές.
 	
 	\textit{Επαγωγική βήμα:} Θα δείξουμε ότι το παραπάνω γράφημα με $n$ κορυφές έχει $2n-3$ ακμές. Έστω $G$ το παραπάνω γράφημα με $n$ κορυφές.Αφαιρώ τη $n-$οστή κορυφή. Για το γράφημα που προκύπτει ισχύει η επαγωγική υπόθεση. Άρα, το $G\backslash u$ έχει $2\cdot (n-1)-3 = 2n-5$ ακμές και επομένως το $G$ που έχει 2 επιπλέον ακμές, θα έχει $2n-5+2=2n-3$ ακμές.
 	
 \begin{figure*}[h!]
	\includegraphics{ex4induction_n.pdf}
	\includegraphics{ex4induction_n_1.pdf}
	\caption{Προσομοίωση επαγωγικού βήματος }
\end{figure*}	

  \end{proof}
  


Αν $r$ το πλήθος των όψεων του $G$ τότε, οι $r-1$ εσωτερικές όψεις είναι τριγωνικές. Μετράω πλήθος ακμών ως προς όψεις. Οι εσωτερικές έχουν 3 ακμές και η εξωτερική είναι κύκλος με $n$ ακμές. Προκύπτει, λοιπόν, ότι
		\[
			3(r-1)+n=2m \Rightarrow r = \cfrac{2m + 2 -n}{3} \overset{\eqref{eq:1}}{\Rightarrow} r = \cfrac{4n-6+2-n}{3} = n-1
		\]
\end{solution}


