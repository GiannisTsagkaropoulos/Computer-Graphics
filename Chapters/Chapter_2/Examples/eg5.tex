\begin{example}
Εκτελέστε περιστροφή $45^\circ$ του τριγώνου $A(0, 0)$, $B(1, 1)$, $C(5, 2)$ 
\begin{enumerate}
    \item[(α)] ως προς την αρχή των αξόνων και 
    \item[(β)] ως προς το σημείο $P(-1, -1)$.
\end{enumerate}
\end{example}


\begin{solution}
Το τρίγωνο περιγράφεται υπό μορφή πίνακα ως εξής:


\[
ABC =
\begin{bmatrix}
0 & 1 & 5 \\
0 & 1 & 2 \\
1 & 1 & 1
\end{bmatrix}
\]

Κάθε στήλη του παραπάνω πίνακα δημιουργείται από τις ομογενείς συντεταγμένες κάθε κορυφής.

α) Για την περιστροφή ως προς την αρχή των αξόνων θα επιδράσουμε πάνω στον $ABC$ με το γνωστό πίνακα $R_{45^\circ}$. Οι συντεταγμένες των κορυφών του καινούργιου τριγώνου προκύπτουν ως εξής:

\[
A'B'C' = R_{45^\circ} \cdot ABC =
\begin{bmatrix}
\cos 45^\circ & -\sin 45^\circ & 0 \\
\sin 45^\circ & \cos 45^\circ & 0 \\
0 & 0 & 1
\end{bmatrix}
\cdot
\begin{bmatrix}
0 & 1 & 5 \\
0 & 1 & 2 \\
1 & 1 & 1
\end{bmatrix}
=
\begin{bmatrix}
\frac{\sqrt{2}}{2} & -\frac{\sqrt{2}}{2} & 0 \\
\frac{\sqrt{2}}{2} & \frac{\sqrt{2}}{2} & 0 \\
0 & 0 & 1
\end{bmatrix}
\cdot
\begin{bmatrix}
0 & 1 & 5 \\
0 & 1 & 2 \\
1 & 1 & 1
\end{bmatrix}
\]

\[
=
\begin{bmatrix}
0 & \frac{\sqrt{2}}{2} & \frac{3\sqrt{2}}{2} \\
0 & \frac{\sqrt{2}}{2} & \frac{7\sqrt{2}}{2} \\
1 & 1 & 1
\end{bmatrix}
\]

Επομένως, οι συντεταγμένες των νέων κορυφών ισούνται με:

\[
A'(0, 0), \quad B'\left(\frac{\sqrt{2}}{2}, \frac{\sqrt{2}}{2}\right), \quad C'\left(\frac{3\sqrt{2}}{2}, \frac{7\sqrt{2}}{2}\right)
\]

β) Εφαρμόζουμε το παράδειγμα 1 για $\theta = 45^\circ$ και $P(-1, -1)$. Ο πίνακας που θα εκτελεί το ζητούμενο μετασχηματισμό θα δίνεται από τη σχέση:

\[
R_{45^\circ, P} = T_{-V} \cdot R_{45^\circ} \cdot T_V
\]

όπου το $V$ στην προκειμένη περίπτωση ισούται με $i + j$. Έτσι:

\[
R_{45^\circ, P} =
\begin{bmatrix}
1 & 0 & -(-1) \\
0 & 1 & -(-1) \\
0 & 0 & 1
\end{bmatrix}
\cdot
\begin{bmatrix}
\frac{\sqrt{2}}{2} & -\frac{\sqrt{2}}{2} & 0 \\
\frac{\sqrt{2}}{2} & \frac{\sqrt{2}}{2} & 0 \\
0 & 0 & 1
\end{bmatrix}
\cdot
\begin{bmatrix}
1 & 0 & -1 \\
0 & 1 & -1 \\
0 & 0 & 1
\end{bmatrix}
\]

\[
=
\begin{bmatrix}
\frac{\sqrt{2}}{2} & -\frac{\sqrt{2}}{2} & -1 \\
\frac{\sqrt{2}}{2} & \frac{\sqrt{2}}{2} & -1 \\
0 & 0 & 1
\end{bmatrix}
\]

Οι συντεταγμένες του καινούργιου τριγώνου προκύπτουν από τη σχέση:

\[
A'B'C' = R_{45^\circ, P} \cdot ABC
\]

\[
\begin{bmatrix}
\frac{\sqrt{2}}{2} & -\frac{\sqrt{2}}{2} & -1 \\
\frac{\sqrt{2}}{2} & \frac{\sqrt{2}}{2} & -1 \\
0 & 0 & 1
\end{bmatrix}
\cdot
\begin{bmatrix}
0 & 1 & 5 \\
0 & 1 & 2 \\
1 & 1 & 1
\end{bmatrix}
=
\begin{bmatrix}
-1 & -\frac{\sqrt{2}}{2} & \frac{3\sqrt{2}}{2} - 1 \\
-1 & \frac{\sqrt{2}}{2} & \frac{9\sqrt{2}}{2} - 1 \\
1 & 1 & 1
\end{bmatrix}
\]

Τελικά οι καινούργιες κορυφές δίνονται από τους τύπους:

\[
A'(-1, -\sqrt{2}-1), \quad B'(-1-\frac{\sqrt{2}}{2}, -1), \quad C'(\frac{3\sqrt{2}}{2}-1, \frac{9\sqrt{2}}{2}-1)
\]

\end{solution}
