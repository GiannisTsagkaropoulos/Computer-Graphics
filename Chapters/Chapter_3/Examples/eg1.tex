\begin{example}
	Να υπολογιστεί ο πίνακας μετασχηματισμού \( T \) για την περίπτωση στροφής γύρω από τον άξονα \( x \) κατά γωνία \( \theta _x \) και στη συνέχεια στροφή ως προς \( y \) κατά γωνία \( \theta _y \).
\end{example}

\begin{solution}
	Ο μετασχηματισμός προκύπτει διαδοχικά. Αρχικά συμβαίνει στροφή γύρω από τον άξονα \( x \) κατά γωνία \( \theta _x \) και έπειτα η στροφή ως προς τον άξονα \( y \) κατά γωνία \( \theta  _y \). Θα εκμεταλλευτούμε τη διαδικάσια σύνθεσης των παραπάνω μετασχηματισμών. Αναλυτικότερα: 
\begin{align*}
	T &= R_{\theta _y, y}  \circ R_{\theta _x, x} =
		\begin{bmatrix}
		\cos \theta_y & 0 & \sin \theta_y & 0 \\
		0 & 1 & 0 & 0 \\
		-\sin \theta_y & 0 & \cos \theta_y & 0 \\
		0 & 0 & 0 & 1
		\end{bmatrix}
	\cdot	
		\begin{bmatrix}
		1 & 0 & 0 & 0 \\
		0 & \cos \theta_x & -\sin \theta_x & 0 \\
		0 & \sin \theta_x & \cos \theta_x & 0 \\
		0 & 0 & 0 & 1
		\end{bmatrix}
	= \\ 
	&=
		\begin{bmatrix}
		\cos \theta_y & \sin \theta_y \sin \theta_x & \sin \theta_y \cos \theta_x & 0 \\
		0 & \cos \theta_x & -\sin \theta_x & 0 \\
		-\sin \theta_y & \cos \theta_y \sin \theta_x & \cos \theta_y \cos \theta_x & 0 \\
		0 & 0 & 0 & 1
		\end{bmatrix}
\end{align*}

\end{solution}